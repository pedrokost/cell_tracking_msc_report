% TO regenerate preample run:
% pdftex -ini -jobname="thesis" "&pdflatex preamble.tex\dump"

\RequirePackage[l2tabu, orthodox]{nag}

\documentclass[a4paper,11pt,twoside,imperial,MSc,declaration=off,titlepagenumber=off]{icldt}

%Uporabljeni paketi
\usepackage[utf8]{inputenc}
\usepackage{cmap}
\usepackage{type1ec}
\usepackage[T1]{fontenc}
\usepackage{fancyhdr}
\usepackage{graphicx,epsfig}
\usepackage[english]{babel}
\usepackage{cite}
\usepackage{caption} % improve caption spacing (among other things)
\usepackage{subcaption}
\usepackage{geometry}
\usepackage[pdftex,colorlinks,citecolor=black,filecolor=black,linkcolor=black,urlcolor=black,pagebackref]{hyperref}
\usepackage[usenames,dvipsnames]{color} % For colors. must before tikz
\usepackage{tikz}
\usepackage[colorinlistoftodos,textsize=small]{todonotes}
\bibliographystyle{ieeetr}
\usepackage{longtable}
\usepackage{tabu}
\usepackage{lipsum}
\usepackage{cleveref}
\usepackage{layout} 
\usepackage{mathrsfs}
\usepackage{amssymb}
\usepackage{mathtools}
\usepackage{epstopdf}  % for compiling pdf to png/jpg during development.
%\usepackage{showframe}
\usepackage{wrapfig}
\usepackage[toc,page]{appendix}
\usepackage{parskip}  % No indent of paragraphs
% \usepackage{dcolumn}  % align columns in table on decimal point

% \newcolumntype{d}[1]{D{.}{\cdot}{#1} }  % for dcolumn

\presetkeys{todonotes}{inline}{}  %fancyline

%Nastavitev glave in repa strani
\pagestyle{fancy}
\fancyhead{}
\renewcommand{\chaptermark}[1]{\markboth{\textsf{Chapter \thechapter:\ #1}}{}}
\renewcommand{\sectionmark}[1]{\markright{\textsf{\thesection\  #1}}{}}
\fancyhead[RE]{\leftmark}
\fancyhead[LO]{\rightmark}
\fancyhead[LE,RO]{\thepage}
\fancyfoot{}
\renewcommand{\headrulewidth}{0.0pt}
\renewcommand{\footrulewidth}{0.0pt}

\newcommand{\gnuplot}{\textbf{gnuplot}}
\newcommand{\pgfname}{\textsc{pgf}}
\newcommand{\tikzname}{Ti\emph{k}Z}

\newcommand{\addref}[1]{\todo[color=red!40]{Cite: #1.}}
\newcommand{\rewrite}[1]{\todo[color=blue!40]{Rewrite: #1}}
\newcommand{\notyetimplemented}[1]{\todo[color=red!40]{This feature has not yet been implemented}}

\newcommand\status[1]{\fbox{\textsc{#1}}}
\newcommand{\statusnew}{\status{new}}
\newcommand{\statusoutline}{\status{outline}}
\newcommand{\statusfirstdraft}{\status{draft I}}
\newcommand{\statusseconddraft}{\status{draft II}}
\newcommand{\statusthirddraft}{\status{draft III}}
\newcommand{\statusproofread}{\status{proofread}}
\newcommand{\statusfinal}{\status{final}}

% Easy coloring
\newcommand\up[1]{{\color{blue}#1}}
\newcommand\dn[1]{{\color{red}#1}}

% Spacing between paragraphs.
\setlength{\parskip}{\baselineskip}
% Spacing between table rows
\renewcommand{\arraystretch}{1.5}

\definecolor{termcolor}{gray}{0.3}
\newcommand{\term}[1]{#1}  % \marginpar{\emph{\raggedright{\textcolor{termcolor}{#1}}}}

\newcommand{\termmargin}[1]{}  % \marginpar{\emph{\raggedright{\textcolor{termcolor}{#1}}}}

\DeclareMathOperator*{\argmax}{arg\,max}

% Tip from: https://www.sharelatex.com/learn/Inserting_Images
% Before producing final report, verert the pdf and jpg, and commend out the epstopdf... line
\epstopdfsetup{outdir=./imageslowres/,update=true}
\epstopdfDeclareGraphicsRule{.eps}{jpg}{.jpg}{%
	convert #1 \OutputFile
}
\DeclareGraphicsExtensions{.jpg,.png,.eps}

\usepackage{microtype}

\author{Pedro Damian Kostelec}
\supervisor{Ben Glocker}
\department{Computing}
\field{Computer Science (Artificial Intelligence)}
\title{Automatic Cell Tracking in Noisy Images for Microscopic Image Analysis}
\date{September 2014}

\def\preambleloaded{Precompiled preamble loaded.}