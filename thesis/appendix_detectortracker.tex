\chapter{Usage guide for the cell detector and tracker}
	\label{app:detectiontracking}
	
	\section{Installation instructions}
	The source code of the cell detector and tracker requires the installation of several dependencies which have to be manually downloaded and added to the MATLAB search path\footnote{\url{http://www.mathworks.co.uk/help/matlab/matlab_env/what-is-the-matlab-search-path.html}}. The dependencies, including the tested version numbers, are:
	
	\begin{enumerate}
	\item {VLFeat}\footnote{\url{http://www.vlfeat.org/}} (version 9.18), an open source library of popular computer vision algorithms.
	\item {svm-struct-matlab}\footnote{\url{http://www.robots.ox.ac.uk/\~vedaldi/code/svm-struct-matlab.html}} (version 1.2), a MATLAB wrapper for $\text{SVM}^\text{struct}$.
	\item The MATLAB code for the inference in the pylon model\footnote{\url{http://www.robots.ox.ac.uk/\~vilem/}}.
	\item {QPBO}\footnote{\url{http://pub.ist.ac.at/\~vnk/software.html}} (version v1.31) from Vladimir Kolmogorov.
	\end{enumerate}
	
	A Bash script is available in the source code in \textit{cell\_tracker/+detector/setup/setup.sh} to automate the installation of the required dependencies. Before running the script it is advisable to review it and configure the installation path in the first few lines of the script.	
	
	The code has been tested in MATLAB R2014a under Ubuntu 14.04.1 LTS. However, the code should work also in version R2013b. Additionally the following MATLAB toolboxes are needed:
	
	\begin{enumerate}
	\item Statistics Toolbox
	\item Image Processing Toolbox
	\item Computer Vision System Toolbox
	\item Neural Network Toolbox
	\item Optimization Toolbox
	\item Parallel Computing Toolbox
	\item System Identification Toolbox
	\end{enumerate}
	
	\section{Usage example}
	
	In this section we describe how to configure the system to train and test the cell detector and tracker on your own image sequences.
	
	First, configure the data directories in the file \textit{dataFolders.m}. This implies appending a block of code similar to this:
	
	
	\begin{lstlisting}[language=Matlab]
	% dataFolders.m
	% ...
	case 7 % dataset ID: use the next number of the sequence
		dotFolder = '<folderNamesWithDotAnnotations>';
		linkFolder =  '<folderNamesWithLinkAnnotations>';
		outFolder = '<outputFolderName>';
		numAnnotatedFrames = 30;         % the number of dot-annotated frames 
		numAnnotatedTrajectories = 4; % the total number of annotated trajectories
	% ...
		        
	\end{lstlisting}
	
	The \textit{dotFolder} should contain images in the \textit{pgm} format named sequentially as \textit{im001.pgm} and correspondingly named \textit{mat} files with dot-annotations for the first frames of the sequence, as indicated by \textit{numAnnotatedFrames}. The \textit{mat} files can be generated by the Image Annotation Tool.
	
	The \textit{linkFolder} should contain images and annotations files in the same format as the \textit{dotFolder}. The \textit{mat} files should not only contain annotated dots but also, at least, a few fully annotated cell trajectories (as indicated by \textit{numAnnotatedTrajectories}) required to train the tracker. Note that in general \textit{dotFolder} and \textit{linkFolder} can be the same.
	
	The \textit{outFolder} is the folder where the detector and tracker will output temporary and final results.
	
	
	
	Second, in the file named \textit{runner.m} it is required to insert the dataset ID as above, and select which of the actions should be executed:
	
	
	\begin{lstlisting}[language=Matlab]
	% runner.m
	% ...
	datasetIDs    = [7];     % Look into dataFolders.m
	
	trainDetector = true;    % Use annotations in dotFolder to train a new detector
	trainTracker  = true;    % Use annotations in linkFolder to train a new tracker
	
	testDetector  = true;    % Evaluate the trained detector on the full image sequence
	testTracker   = true;    % Evaluate the trained tracker on the full image sequence
	
	showTracks    = true;    % Display a figure containing the generated trajectories
	% ...
       
	\end{lstlisting}
	
	Note that it is possible to insert several dataset IDs and the code will be executed on each of them sequentially. The file \textit{runner.m} can then be executed using MATLAB.
	
