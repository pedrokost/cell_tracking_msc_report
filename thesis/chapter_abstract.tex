\begin{abstract}
	Automatic cell detection and tracking for \textit{in vivo} microscopic image sequences enables the efficient analysis and quantification of cell behaviour in their natural environment, thus improving the study of drugs and our understanding of living beings. The technical achievements in \textit{in vivo} image acquisition in past few years enable us to study cell behaviour in a much more detailed way without dramatically changing their natural environment. This research focused on the detection and tracking of cells in these images, which often exhibit motion artefacts, blurred frames, obscured frames and background noise. Traditional cell detection methods cannot efficiently analyse these images. 
	
	We approach the problem from a data point of view. We upgraded a previous cell detection method that learns to classify non-overlapping candidate regions as cell or not-cell. The computation time of the method has been significantly cut to process each frame (of dimensions around 400-by-400 pixels) within 0.5 and 1.5 seconds. The method achieves high precision and recall values on the studied datasets and can be trained with dot-annotation -- each dot corresponding to a cell. In some cases it is possible to use a pre-trained detector to detect cells in a new, previously unseen dataset and still achieve good detection rates.
	
	The tracking method uses the centroids of the detected cells and attempts to link them into trajectories. The method relies on a global data association approach to reliably generate trajectories based on a global decision. This research has upgraded previous methods to rely on a large set of features obtained from observed data. A machine learning approach is used to compute likelihoods for linking cell detections into increasingly longer trajectories. Because the likelihoods are obtained directly from training examples, the method is able to produce good results also on noisy datasets, where several frames in a sequence can come out-of-focus, cells disappear and reappear over time, etc.
	
	The combinations of these two data driven algorithms for cell detection and tracking has been shown to be a promising approach for automatic cell tracking. Extensive evaluation has shown that the methods achieve very good results on the studied datasets. We also present a set of possible further improvements to the methods that would further improve their performance. 
\end{abstract}
