\begin{abstract}
	Automatic cell detection and tracking for \textit{in vivo} microscopic image sequences enables the efficient analysis and quantification of cell behaviour in their natural environment, thus improving the study of drugs and our understanding of living beings. The technical achievements in \textit{in vivo} image acquisition in the past few years have enabled us to study cell behaviour in a much more detailed manner without dramatically changing their natural environment. This research focused on the detection and tracking of cells in microscopic images that often exhibit motion artifacts, blurred frames, obscured cells and background noise. In fact, by using current cell detection and tracking methods, it would not have been possible to efficiently analyse these images.
	
	To achieve this, the author improved a cell detection method that learns to classify non-overlapping candidate regions as cell or not-cell. The computation time of the original method has been significantly reduced to process each frame (of dimensions around 512-by-512 pixels) between 0.5 and 1.5 seconds. This method achieves high precision and recall values on the studied datasets and can be trained with dot-annotations -- each dot corresponding to a cell. In some cases it is possible to use a pre-trained detector to detect cells in a new, previously unseen dataset and still achieve good detection rates.
	
	The author also developed a tracking method that attempts to link the centroids of the detected cells into trajectories. The method relies on a global data association approach to reliably generate trajectories based on a global decision. Within the context of this research, previous methods have been improved by approaching the problem of defining likelihoods of linking tracklets from a machine learning point of view. Because the likelihoods are obtained directly from training examples, it was possible to produce good results even on low quality images, where several frames in a sequence can become out-of-focus, and in which cells can disappear and reappear over time, etc.
	
	Overall, the combination of these two data driven algorithms for cell detection and tracking has been shown to be a promising approach to automatic cell tracking. The system has shown good results on the studied datasets. Within the scope of this dissertation, the author has also presented various possible additional improvements to the methods, which, if applied, might even further improve their performance. 
\end{abstract}
