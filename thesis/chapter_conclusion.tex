		
\chapter{Conclusions and future work}
\label{chap:conclusion}

In this final chapter of the report we present some concluding remarks and enumerate a list of possible upgrades to improve the cell detection and tracking modules.

\section{Conclusion}
\label{sec:conclusion_conclusion}

Improved microscopy imaging techniques allow us to gather large amounts of cell microscopy images. The manual analysis of these images would be an error prone and slow process, requiring days of manual work to review some hundreds of frames. The advances of computer vision algorithms for cell detection and tracking over the past decades magnified by the increased computational power of modern computers allow for an efficient analysis of these datasets in a fraction of the time compared to manual analysis.

The large amount of data that can be analysed with these new methods improve the quality of cell research. They allow for new insights into drug development and a better understanding of the living body. Specifically, this research was focused on enabling the efficient analysis of neutrophils which have a crucial role in the clearance of infections. Their careful analysis could help explain their prominent presence in certain organs, such as the lung. It could also help discover any additional activities that these leukocytes perform, and clarify whether they develop from a single or several neutrophil predecessors.

In this three month project we have identified a pipeline of algorithms that enables the automated analysis of neutrophil behaviour in sometimes noisy images of varying contrasts. This required identifying a robust algorithm to detect cells in these images, and develop a tracking method that would perform well with imperfect cell segmentation and a certain amount of missed detections.

We have upgraded a cell detector developed by Arteta \emph{et al.}~\cite{arteta12}. The detector was able to learn how to discriminate between candidate cell regions as either cell or not-cell. We have successfully applied the method to our datasets and improved its speed to make it usable for detecting cells in hundreds of frames. The method was able to robustly detect cells (albeit with some false positive and false negatives) after being trained with a small number of dot-annotated images.

We have developed a tracking method inspired by Bise \emph{et al.}~\cite{bise11global} that performs a global decision to join robust tracklets into longer ones. We have modified the original method to heavily rely on the input data, thus eliminating the need for a number of heuristics, which would likely have made the algorithm perform worse when presented with a new dataset. The new approach only requires the algorithm to be retrained using a small number of annotated trajectories. Although the method is automatic, the user is presented with four parameters, which can be adjusted to improve the generated trajectories. Experiments designed to test the performance of the method have shown promising results.

We have also developed efficient image annotation tools to annotate images with dots and links connecting them. These tools can be used for the annotation of any point-like objects and include features specifically designed to increase the clarity of noisy and low contrast images to facilitate the annotation. Furthermore, the annotation tool can be used to review the cell detection results.

Overall, the detection method was able to detect cell detections with high precision and recall values, thus enabling the tracking method to effectively connect the cell detection results and link robust tracklets into longer trajectories by closing gaps of up to 20 frames.

\section{Future work}
\label{sec:conclusion_futurework}

The work developed in this thesis is promising in the manner in which it can deliver automatic cell detection and tracking; however, the method can be further improved, and alternative methods researched. Below we present a list of possible improvements that would likely make the algorithm more robust.

The process of obtaining cell images \textit{in vivo} is challenging especially in moving organs such as the lung, where the motion of the tissue causes the images to jiggle or lose focus. The jiggling can be eliminated using a pre-processing step that would stabilize the images using information hidden in the background, such as blood vessels. This would result in smoother trajectories. Furthermore, this would simplify the computations of spatio-temporal features to train the cell tracker module, as it would be easier to predict the velocity of the cells.

This research was focused on tracking cells in order to facilitate the analysis of cell behaviour. The better suit this goal, it would be beneficial if the system returned a profile for each tracked cell including their appearance and mode of motion. The cell detection module did not focus on an accurate segmentation of the cells. Upgrading the system to accurately segment the cells after they have been identified would not only provide an appearance profile of the cells, but also improve the tracking system. 

From the datasets we analysed, the original images for dataset D (described in \cref{sec:datasetD}) include a large portion that are completely unusable for the tracker because the cells were blurred or invisible for a large number of frames. These frames have been manually removed. It could be beneficial to automate this process by automatically detecting images that are of too low a quality to be usable and subsequently discard of them, whilst leaving a mark with the number of frames skipped. If this step could be performed quickly, the total computation time would be reduced as such frames wouldn't need to be analysed by the cell detector.

The accuracy of the tracking module is heavily dependent on the quality of features that are computed for the tracklets. We have implemented and tested a broad range of spatio-temporal features, including a linear Kalman filter. Instead of assuming that the tracklet's velocity would be linear with respect to the last few frames of a trajectory, it would be beneficial to use a model that would be able to predict the cell direction more accurately. For example, an interacting multiple models motion filter running several Kalman filters in parallel has been proved to better predict future cells positions \cite{li08}.

In \cref{sec:tracking_implementation} we mentioned the difficulty of computing spatio-temporal features for robust tracklets with a single cell observation. For this reason, they have not been considered as candidate linking tracklets. However, these short tracklets include information that could improve the detail of generated trajectories. Further research into how to best use these short tracklets could results in trajectories that more accurately conform to the real cell motion.

The developed tracking method has been tested on hundreds of frames of microscopy images. However, as the number of frames is increased, the method is likely to reach a bottleneck due to memory usage. In order to improve the space requirements of the tracker and permit the tracking of thousands of image frames, it would be beneficial to ensure that the processing of tracklets is performed in windows, a few hundred frames at a time. Thus the tracker would first generate tracklets within each window, and then link these tracklets between windows. 

In terms of user experience, a simple graphic user interface to load the image sequences and start the tracking process would greatly improve the approachability of the method to a larger non-technical audience. A more advanced user interface could also include options to train and evaluated a new model.

Finally, whilst the above improvements relate to the software, it is expected that the imaging technique will also improve. This could alleviate the problem of jiggling cells and out of focus images, thus reducing the need to overcome these imaging limitations in the software.