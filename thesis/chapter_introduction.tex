	\chapter{Introduction \status{new}}

	\section{Motivation \status{new}}
		\todo[inline]{To develop methods for cell tracking}
		
		Recent advances in intravital microscopy enable us to study the behaviour of different cells without excessively modifying the natural environment in which these cells are found within the body of mice.
		
		
		\rewrite{It is too similar to the paper}
		Neutrophiles, a type of leukocytes, are a major player during acute inflammation. The presence of neutrophiles in the site of inflammation is crucial for the resolution of an infection \cite{kolku13}. A significant reduction of neutrophiles within the human body can lead to severe immunodeficiency or death. A fatal outcome is also the result of the absence of these leukocytes in mice.
		
		Neutrophiles can be found in the liver, bone marrow, spleen and, in larger quantities, the lung. The observation of neutrophil behaviour within these organs should help explain their large presence within these organs. 
		
		Recent data shows that their functions extends beyond the clearance of infections. A recent study \cite{pillay10} has revealed that their half-life span is much longer than previously believed (up to 12.5 hours for mouse cells and 5.4 days in humans). Although the results of this study are doubtful \cite{toft11} their longevity is increased during their activation during inflammation. This longer lifespan may allow neutrophiles to perform a wider range of complex functions in a tissue, beyond the clearance of infections. Direct, real-time observation of their behaviour may help reveal these functions and extend our understanding of the life of a neutrophil.
		
		Finally, there is accumulating evidence to support the existence of distinct subset of neutrophiles with discrete roles in infection, inflammation and cancer immunology. However, it is unclear if these are truly distinct lineages or instead develop from the same neutrophil precursor. The observation of the large quantities of neutrophil data should help support or abandon \rewrite{abandon} the hypothesis of multiple separate lineages.
		
	\section{Objectives \status{new}}
		\todo[inline]{Describe the different modules of the system - detector, tracker, stat computation}
		
		- create a robust pipeline for tracking neukocytes in image sequences of noisy images obtained throught that ventilator method
		
		- the system should be able to accept an image sequence, detect the cell in each frame, and uset hem to compute the trajectories of the cells.
		
		- A basic system to return metrics about the trajectories
		
	\section{Contributions \status{new}}
		\todo[inline]{Machine learning based cell tracker system} that is able to work on highly noisy datasets, where several frames can come out of focus, cells dissapear and reapear, etc.
		\todo[inline]{Dot Annotation GUI}
		to ease the annotation of datasets of point objects, as well as link them to create trajectories in a frame by frame basis
		\missingfigure{Make a sketch}
	\section{Thesis structure \status{new}}
		The remained of the thesis is structured as follows. 
		
		chapter 2 is a brief literature survey outlining the exesting methods for cell detection and tracking
		
		chapter 3 describes the cell detection module and its implementation
		
		chapter 4 describes the tracking module
		
		chaapter 5 represents the biological statistics module, and we focus on listing the different types of metrics that it computes
		
		chapter 6 represents the image annotation tool that was created to facilitate image annoations
		
		chapter 7 evaluates the cell detector and cell tracker
		
		chapter 8 is reserved for some concluding remarks and ideas that could be used to enhance this work.