	\chapter{Introduction}


	In this introductory chapter we describe the problem of tracking cells in microscopy images manually and motivate the development of an automatic cell tracker. We also outline the objectives and contributions of the project and provide a report outline for the following chapters.
	
	\section{Motivation}
		
		Recent advances in intravital microscopy enable us to study the behaviour of different cells without excessively modifying the natural environment in which these cells are found within the observed organism. 
			
		Neutrophils are a type of leukocyte that have a crucial role in the clearance of infections \cite{kolku13}. A significant reduction in the number of neutrophils in the human body (or in mice) leads to severe immunodeficiency or death. They can be found in the bone marrow, liver, spleen, lung and throughout the circulatory system. Direct observation of neutrophils should help explain their function in these organs. 
		
		Of special interest is the transit of neutrophils in the fine and specialized capillary network located in the lung. To transit, the neutrophils must come in contact with the endothelium of these small vessels. In order to protect against pathogens, neutrophils are potently cytotoxic. Therefore, their regulation in the lung is important as the consequences of their unwarranted activation would be severe.
		
		A recent study \cite{pillay10} has shown that the lifespan of neutrophils is much longer than previously predicted (up to 12.5 hours for mice and 5.4 days for humans). Although this specific study is a source of doubtful criticism \cite{toft11}, the longevity of neutrophils increases during inflammation. This longer lifespan may permit them to perform a wider range of complex activities, beyond the clearance of infections. An analysis of their behaviour as observed through intravital microscopy could help reveal more of their roles.
		
		Finally, there is accumulating evidence to support the existence of different lineages of neutrophils with discrete roles. Automated analysis of microscopic images could explain whether these are truly distinct lineages or if they instead all develop from the same neutrophil predecessor.
		
		The observation of neutrophils requires the analysis of hundreds of frames of microscopy image sequences. The manual annotation of these image sequences in order to extract trajectories of movement of neutrophils is a time consuming and error prone process. Manual annotation severely limits the amount of data that can be analysed and slows down the advancement of cell research. It would be a major advance to be able to rely on the automatic identification and tracking of cells over time in sometimes noisy complex images.
		
	\section{Objectives}
		
		The clarity of neutrophil image sequences obtained \textit{in vivo} can vary considerably due to the combination of the raster scanning imaging technique and, in the case of the lung, the motion of the tissue induced by a mechanical ventilator. In the images we can observe artifacts such as out-of-focus frames and blurred images due to the jiggling tissue, etc. This can sometimes make it difficult to identify and segment individual cells.
		
		The aim of this research is to develop methods for cell detection and tracking. This system should be able to accept image sequences of cells in tissue, identify the cells, and track their position over the entire sequence.
		
		The identification of cells should take into account the nature of the imaging technique and the quality of the studied images. Simple methods such as thresholding would be too unreliable; more advanced methods need to be investigated.
		
		Tracking of cells should also take into account the nature of input data. Basic frame-by-frame tracking of cells is likely to have poor results because the cells frequently disappear into the depths of tissue or lose focus. Methods that take into account the temporal behaviour of cells will probably produce better results.
		
	\section{Contributions}
		
		The work in this project led to the development of a pipeline of algorithms for automatic cell tracking. The pipeline combines a cell detection algorithm from \cite{arteta12} and, to the best knowledge of the author, a novel tracking method that is directly based on the observed data. 
		
		The cell tracking module uses a global data association approach to reliably generate cell detection trajectories based on a global decision. This research has improved previous methods allowing them to rely on a large set of features obtained from observed data. A machine learning approach is used to compute likelihoods for linking cell detections into increasingly longer trajectories. Because the likelihoods are obtained directly from training examples, the method is able to produce good results even from noisy datasets, where several frames in a sequence can become out-of-focus, and in which cells can disappear and reappear over time, etc.
		
		Finally, an image annotation tool is developed that allows simple dot annotation of cells and linking of cells among frames to represent trajectories. This tool can be used for generating the necessary training data and further allows to visualise and inspect the results produced by the automatic cell tracking method.
		
	\section{Report structure}
		The rest of the thesis is structured as follows:
		
		\Cref{chap:relatedwork} consists of brief literature survey outlining existing methods for cell detection and tracking. The chapter gives a background of how the methods for cell detection and tracking have evolved over time.
		
		\Cref{chap:cell_detection} describes in more depth the cell detector from Arteta \emph{et al.}~\cite{arteta12} and outlines the modifications that improved the detection speed.
		
		\Cref{chap:tracking} describes in detail the cell tracking module. It outlines the two step process of generating robust tracklets and then joining them into long trajectories using trained classifiers.
	
		\Cref{chap:data} represents an overview of example datasets upon which the tracking tool is tested and presents a cell annotation tool developed to ease the annotation of image sequences.
		
		\Cref{chap:results} quantitatively and qualitatively evaluates the performance of the cell detection and tracking modules. Experiments used to test the methods and their results are provided.
		
		\Cref{chap:conclusion} includes some concluding remarks and ideas that could be implemented to continue the advancement in the field of automatic cell detection and tracking.